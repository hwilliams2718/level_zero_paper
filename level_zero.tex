\documentclass[11pt]{amsart}
\usepackage{mathtools,amsmath,amsthm,amssymb,amsbsy,amstext,amsopn}
\usepackage{xcolor}
\usepackage{graphicx}
\usepackage{microtype}
\usepackage[centering,margin=1.2in]{geometry}
\usepackage[colorlinks,
            linkcolor=black!50!red,
            citecolor=blue,
            pdfpagemode=None]{hyperref}
\usepackage{cleveref}

\usepackage[draft]{say}
\newcommand{\sayHW}[1]{\say[HW]{\color{violet}{\bf HW:}\;#1}}
\newcommand{\saySS}[1]{\say[SS]{\color{blue}{\bf VS:}\;#1}}
\newcommand{\sayDR}[1]{\say[DR]{\color{red}{\bf EZ:}\;#1}}

\begin{document}
\title{Regular Representations of Affine Quivers and Level Zero Representations of Affine Lie Groups}

\newtheorem{theorem}{Theorem}[section]
\newtheorem{conjecture}[theorem]{Conjecture}
\newtheorem{corollary}[theorem]{Corollary}
\newtheorem{definition}[theorem]{Definition}
\newtheorem{lemma}[theorem]{Lemma}
\newtheorem{proposition}[theorem]{Proposition}
\newtheorem{example}[theorem]{Example}
\newtheorem{remark}[theorem]{Remark}

\author{Harold Williams}
\address{Harold Williams\newline
University of Texas at Austin\newline
Department of Mathematics\newline
Austin TX 78712\newline
USA}
\email{hwilliams@math.utexas.edu}

\author[Rupel]{Dylan Rupel}
\address[Dylan Rupel]{University of Notre Dame}
\email{drupel@nd.edu}

\author[Stella]{Salvatore Stella}
\address[Salvatore Stella]{Universit\`a degli studi di Roma ``La Sapienza''}
\email{stella@mat.uniroma1.it}

\begin{abstract}
\end{abstract}

\maketitle

\section{Introduction}

\begin{theorem}\label{thm:maintheorem}
Let $G$ be an affine Lie group of type $A_n^{(1)}$, $c$ a Coxeter element of the affine Weyl group, and $Q_c$ the associated affine quiver.  The coordinate ring of $G^{c,c^{-1}}$ is an upper cluster algebras of type $Q_c$ and all unfrozen cluster variables are restrictions of principal minors of representations of $G$ with finite-dimensional weight spaces.  Those that are cluster characters of preprojective, regular, or preinjective representations of $Q_c$ are minors of positive level, level zero, and negative level $G$-representations, respectively.
\end{theorem}

\begin{conjecture}\label{conj:mainconjecture}
The above theorem holds for arbitrary affine types.
\end{conjecture}

\section{Cluster Structures on Coxeter Double Bruhat Cells}
\section{Cluster Variables of Affine Quivers}

\section{Highest and Lowest Weight Minors}

\begin{proposition}
\label{prop:fundid}
Suppose $u, v \in W$, $k \in [1,n]$ satisfy $u s_k \omega_k < u \omega_k$ and $v s_k \omega_k < v \omega_k$.  Then \sayHW
\[
\Delta_{v \omega_k}^{u \omega_k}\Delta_{v s_k \omega_k}^{u s_k \omega_k} = 
\Delta_{v \omega_k}^{u s_k \omega_k}\Delta_{v s_k \omega_k}^{u \omega_k} + 
\prod_{i \neq k}
\]
\end{proposition}

\begin{lemma}
For $k \in [1,n]$, $m \geq 1$, the identity
\begin{equation}
\Delta_{c^{m-1}\omega_k}^{c^{m-1}\omega_k}\Delta_{c^{m}\omega_k}^{c^{m}\omega_k} = \Delta_{c^{m}\omega_k}^{c^{m-1}\omega_k}\Delta_{c^{m-1}\omega_k}^{c^{m}\omega_k} + \left(\prod_{i < k}(\Delta_{c^{m}\omega_i}^{c^{m}\omega_i})^{-a_{ik}}\right)\left(\prod_{i>k}(\Delta_{c^{m-1}\omega_i}^{c^{m-1}\omega_i})^{-a_{ik}}\right)
\end{equation}
holds on $G^{c,c^{-1}}$.
\end{lemma}
\begin{proof}

\end{proof}

\begin{proposition}
Let $G$ be a Kac-Moody group, $c$ a Coxeter element of its Weyl group, and $B_c$ the associated exchange matrix.  The coordinate ring of the double Bruhat cell $G^{c,c^{-1}}$ is an upper cluster algebra of type $B_c$ with coefficients.  The coordinate ring of the reduced double Bruhat cell $L^{c,c^{-1}}$ is an upper cluster algebra of type $B_c$ with principal coefficients.
\end{proposition}
\begin{proof}
\begin{enumerate}
\item Cite other papers.
\item Generalities on reducing double Bruhat cells and forgetting cefficients.
\item Redo Yang-Zelevinsky change of coefficients.
\end{enumerate}
\end{proof}

\begin{proposition}
Let $G$ be a Kac-Moody group, $c$ a Coxeter element of its Weyl group, and $B_c$ the associated exchange matrix.  All initial and preprojective cluster variables in the coordinate ring of $G^{c,c^{-1}}$ are restrictions of principal minors of highest weight representations.  Their $g$-vectors are of the form $\dotsc$, and the cluster variable with $g$-vector $g = (g_1,\dotsc,g_n)$ is the restriction of the principal minor of weight $\sum g_i \omega_i$.
\end{proposition}
\begin{proof}
\begin{enumerate}
\item Redo Yang-Zelevinsky computation of the exchange relations among preprojective minors.
\end{enumerate}
\end{proof}

\begin{proposition}
Let $G$ be a Kac-Moody group, $c$ a Coxeter element of its Weyl group, and $B_c$ the associated exchange matrix.  All preinjective cluster variables in the coordinate ring of $G^{c,c^{-1}}$ are restrictions of principal minors of lowest weight representations.  Their $g$-vectors are of the form $\dotsc$, and the cluster variable with $g$-vector $g = (g_1,\dotsc,g_n)$ is the restriction of the principal minor of weight $\sum g_i \omega_i$.
\end{proposition}
\begin{proof}
\begin{enumerate}
\item Compute change of coordinates between oppositely shuffled parametrizations.
\item Use this to show antifundamental minors are mutations of fundamental ones.
\item Think about coefficients in oppositely shuffled parametrization.
\item Show negative version of generalized minor identities reduce to exchange relations among preinjective minors.
\end{enumerate}
\end{proof}

\section{Level Zero Representations}

\begin{lemma}
Let $c$ be a Coxeter element of the affine Weyl group of type $A_n^{(1)}$ and $Q_c$ the associated quiver.  For $[p,q]$ a cyclic interval in $[0,n]$, the $Q_c$-representation $M_{[p,q]}$ is regular if and only if the total number of sinks and sources in $[p,q]$ is even.
\end{lemma}
\begin{proof}
\begin{enumerate}
\item Show that the parity condition is preserved under mutation.
\end{enumerate}
\end{proof}

\begin{proposition}
Let $G = \widehat{LSL_{n+1}}$ be the affine Kac-Moody group of type $A_n^{(1)}$, $c$ a Coxeter element of the affine Weyl group, and $Q_c$ the associated affine quiver.  All regular cluster variables in the coordinate ring of $G^{c,c^{-1}}$ are principal minors of level zero representations.  
\end{proposition}
\begin{proof}
\begin{enumerate}
\item Given a regular representation $M_{[p,q]}$ compute the strand configuration whose minor has the same $F$-polynomial.
\item Show that the injective copresentation matches the weight of the minor.
\end{enumerate}
\end{proof}

\section{Examples in Other Types}

\begin{proposition}
\Cref{conj:mainconjecture} holds when $G$ is of type $D_4^{(1)}$, $B_2^{(1)},\dotsc$ and $c$ is $\dotsc$. 
\end{proposition}
\begin{proof}
\begin{enumerate}
\item Write out the computation.
\end{enumerate}
\end{proof}

\section{Appendix: Double Bruhat Cells}
\section{Appendix: Cluster Algebras}
\section{Appendix: Quiver Representations}

\section{Conventions}

\begin{itemize}
\item The minor $\Delta_\omega^\delta$ is our notation for YZ's $\Delta_{\delta, \omega}$.
\item The Cartan matrix is $a$ and it's $n \times n$.
\end{itemize}

\end{document}
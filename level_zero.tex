\documentclass[11pt]{amsart}
\usepackage{mathtools,amsmath,amsthm,amssymb,amsbsy,amstext,amsopn}
\usepackage{xcolor}
\usepackage{graphicx}
\usepackage{microtype}
\usepackage[centering,margin=1.2in]{geometry}
\usepackage[colorlinks,
            linkcolor=black!50!red,
            citecolor=blue,
            pdfpagemode=None]{hyperref}

\begin{document}
\title{Regular Representations of Affine Quivers and Level Zero Representations of Affine Lie Groups}

\author{Harold Williams}
\address{Harold Williams\newline
University of Texas at Austin\newline
Department of Mathematics\newline
Austin TX 78712\newline
USA}
\email{hwilliams@math.utexas.edu}

\author[Rupel]{Dylan Rupel}
\address[Dylan Rupel]{University of Notre Dame}
\email{drupel@nd.edu}

\author[Stella]{Salvatore Stella}
\address[Salvatore Stella]{Universit\`a degli studi di Roma ``La Sapienza''}
\email{stella@mat.uniroma1.it}

\begin{abstract}
\end{abstract}

\maketitle

\section{Introduction}

\begin{theorem}
Let $G$ be an affine Lie group of type $A_n^{(1)}$, $c$ a Coxeter element of the affine Weyl group, and $Q_c$ the associated affine quiver.  The coordinate ring of $G^{c,c^{-1}}$ is an upper cluster algebras of type $Q_c$ and all unfrozen cluster variables are restrictions of principal minors of representations of $G$ with finite-dimensional weight spaces.  Those that are cluster characters of preprojective, regular, or preinjective representations of $Q_c$ are minors of positive level, level zero, and negative level $G$-representations, respectively.
\end{theorem}

\begin{conjecture}
The above theorem holds for arbitrary affine types.
\end{conjecture}

\section{Cluster Structures on Coxeter Double Bruhat Cells}
\section{Cluster Variables of Affine Quivers}

\section{Highest and Lowest Weight Minors}
\section{Level Zero Representations}
\section{Examples in Other Types}

\section{Appendix: Double Bruhat Cells}
\section{Appendix: Cluster Algebras}
\section{Appendix: Quiver Representations}

\end{document}
\documentclass[11pt]{amsart}
\usepackage{mathtools,amsmath,amsthm,amssymb,amsbsy,amstext,amsopn}
\usepackage{xcolor}
\usepackage{graphicx}
\usepackage{microtype}
%\usepackage[centering,margin=1.2in]{geometry}
\usepackage[colorlinks,
            linkcolor=black!50!red,
            citecolor=blue,
            pdfpagemode=None]{hyperref}
\usepackage{cleveref}

\usepackage[draft]{say}
\newcommand{\sayHW}[1]{\say[HW]{\color{violet}{\bf HW:}\;#1}}
\newcommand{\saySS}[1]{\say[SS]{\color{blue}{\bf SS:}\;#1}}
\newcommand{\sayDR}[1]{\say[DR]{\color{red}{\bf DR:}\;#1}}

% shorthands 
\newcommand{\cA}{\mathcal{A}}
\newcommand{\bC}{\mathbb{C}}
\newcommand{\cAb}{\mathcal{A}_\bullet}
\newcommand{\rep}{\operatorname{rep}}

% ambients
\newtheorem{theorem}{Theorem}[section]
\newtheorem{conjecture}[theorem]{Conjecture}
\newtheorem{corollary}[theorem]{Corollary}
\newtheorem{definition}[theorem]{Definition}
\newtheorem{lemma}[theorem]{Lemma}
\newtheorem{proposition}[theorem]{Proposition}
\newtheorem{example}[theorem]{Example}
\newtheorem{remark}[theorem]{Remark}

\begin{document}
\title{Regular Representations of Affine Quivers and Level Zero Representations of Affine Lie Groups}

\author{Harold Williams}
\address[Harold Williams]{
University of Texas at Austin\newline
Department of Mathematics\newline
Austin, TX 78712\newline
USA}
\email{hwilliams@math.utexas.edu}

\author[Rupel]{Dylan Rupel}
\address[Dylan Rupel]{
University of Notre Dame\newline
Department of Mathematics\newline
Notre Dame, IN 46556\newline
USA}
\email{drupel@nd.edu}

\author[Stella]{Salvatore Stella}
\address[Salvatore Stella]{Universit\`a degli studi di Roma ``La Sapienza''}
\email{stella@mat.uniroma1.it}

\begin{abstract}
\end{abstract}

\maketitle

\section{Introduction}

\begin{theorem}\label{thm:maintheorem}
Let $G$ be an affine Lie group of type $A_n^{(1)}$, $c$ a Coxeter element of the affine Weyl group, and $Q_c$ the associated affine quiver.  The coordinate ring of $G^{c,c^{-1}}$ is an upper cluster algebras of type $Q_c$ and all mutable cluster variables are restrictions of principal minors of representations of $G$ with finite-dimensional weight spaces.  Those that are cluster characters of preprojective, regular, or postinjective representations of $Q_c$ are minors of positive level, level zero, and negative level $G$-representations, respectively.
\end{theorem}

\begin{conjecture}\label{conj:mainconjecture}
The above theorem holds for arbitrary affine types.
\end{conjecture}

\section{Cluster Structures on Coxeter Double Bruhat Cells}

\section{Highest and Lowest Weight Minors}

Throughout this section we fix a symmetrizable Kac-Moody group $G$ of infinite type and a Coxeter element $c$.  

\begin{proposition}
The coordinate ring of $G^{c,c^{-1}}$ is an upper cluster algebra of type $B_c$ with doubled principal coefficients, that is whose $3n \times n$ exchange matrix has principal part $B_c$ and frozen part two copies of the identity matrix.  The initial cluster variables are the principal minors $\Delta_{\omega_i}^{\omega_i}$.  The frozen variables are
\[
y_j = \Delta^{\omega_j}_{ c\omega_j} \prod_{i < j}(\Delta^{\omega_i}_{c \omega_i})^{a_{ij}}, \quad z_j = \Delta^{c \omega_j}_{\omega_j} \prod_{i < j}(\Delta^{c \omega_i}_{\omega_i})^{a_{ij}}
\]
\end{proposition}
\begin{proof}
Let $B'$ be the $3n \times n$ exchange matrix with principal part is $B_c$ and with $B'_{i+n,j} = B'_{i+2n,j} = [i = j] + a_{ij}[i < j]$ for $i,j \in [1,n]$.  Then the coordinate ring of $G^{c,c^{-1}}$ is the upper cluster algebra with exchange matrix $B'$, initial cluster variables $\Delta_{\omega_i}^{\omega_i}$, and frozen variables $\Delta^{\omega_i}_{c \omega_i}$, $\Delta^{c \omega_i}_{\omega_i}$ \cite{me}.  
\end{proof}

\begin{proposition}\label{prop:fundid}
Suppose $u, v \in W$, $k \in [1,n]$ satisfy $u s_k \omega_k < u \omega_k$ and $v s_k \omega_k < v \omega_k$.  Then the identity \sayHW{May make sense to just state on the derived subgroup, so we can ignore ``extended'' fundamental weights.}
\[
\Delta_{v \omega_k}^{u \omega_k}\Delta_{v s_k \omega_k}^{u s_k \omega_k} = 
\Delta_{v \omega_k}^{u s_k \omega_k}\Delta_{v s_k \omega_k}^{u \omega_k} + 
\prod_{i \neq k} (\Delta_{v\omega_i}^{u \omega_i})^{-a_{ik}}
\]
holds on $G$.
\end{proposition}

For any root $\alpha$ we define $[\alpha:\alpha_i]$ by $\alpha = \sum_{i=1}^n [\alpha:\alpha_i]$, defining $[\alpha:\beta_k^+]$ analogously.

\begin{lemma}\label{lem:minorsondbc}
Restricted to $G^{c,c^{-1}}$ we have the generalized minor identity
\begin{equation}
\Delta_{c^{m-1}\omega_k}^{c^{m-1}\omega_k}\Delta_{c^{m}\omega_k}^{c^{m}\omega_k} = 
\prod_{j=1}^n \left( \Delta^{c \omega_j}_{\omega_j}\Delta^{\omega_j}_{ c\omega_j} \prod_{i < j}(\Delta^{c \omega_i}_{\omega_i} \Delta^{\omega_i}_{c \omega_i})^{a_{ij}}\right)^{[c^{m-1} \beta_k^+:\alpha_j]} + 
\left(\prod_{i < k}(\Delta_{c^{m}\omega_i}^{c^{m}\omega_i})^{-a_{ik}}\right)\left(\prod_{i>k}(\Delta_{c^{m-1}\omega_i}^{c^{m-1}\omega_i})^{-a_{ik}}\right)
\end{equation}
for any $k \in [1,n]$, $m \geq 1$.  
\end{lemma}
\begin{proof}
The proof consists in comparing the given expression with the $u = v = c^{m-1}s_1 \cdots s_{k-1}$ case of \Cref{prop:fundid}.  We first claim that restricted to $G^{c,c^{-1}}$ we have
\begin{equation}\label{eq:coeffreduction}
\Delta_{c^m \omega_k}^{c^{m-1} \omega_k} = 
\prod_{i=1}^n (\Delta_{c \omega_i}^{\omega_i})^{[c^{m-1}\beta_k^+:\beta_i]}.
\end{equation}
 It suffices to evaluate both sides on a generic $g \in G^{c,c^{-1}}$ of the form
\[
g = hx_{-1}(t_{-1}) \cdots x_{-n}(t_{-n}) x_{n}(t_{n}) \cdots x_{1}(t_{1})
\]
for $h \in H$ and $t_i \in \bC^*$.  Since $(s_1 \cdots s_n)^m$ is a reduced word for $c^m$ for any $m \geq 1$\sayHW{citation needed?}, a straightforward calculation with root strings in the highest weight representation $L_{\omega_i}$ yields
\begin{equation}\label{eq:coeffeval}
\Delta_{c^m \omega_i}^{c^{m-1}\omega_i}(g) = h^{c^{m-1}\omega_i} \prod_{j=1}^n t_j^{[c^{m-1}\beta_i^+:\alpha_j]}
\end{equation}
We can rewrite this using the fact that
\[
c^{m-1} \omega_i = \sum_{j=1}^n [c^{m-1}\beta_i^+:\beta_j^+]\omega_j.
\]
To see this first recall that the $\beta_i^+$ are linearly independent, as they differ from the $\alpha_i$ by a unitriangular transformation.  Thus $1-c$ is an isomorphism from the subspace spanned by the $\omega_i$ to that spanned by the $\beta_i^+$, and the identity follows since $1-c$ commutes with $c^{m-1}$.  Using this and the immediate identity
\[
[\alpha:\alpha_j] = \sum_{i=1}^n [\alpha:\beta_i^+][\beta_i^+:\alpha_j],
\]
we can rewrite the $i = k$, arbitrary $m$ case of \cref{eq:coeffeval} in terms of the $m = 1$, arbitrary $i$ case and obtain \cref{eq:coeffreduction}.

Since $\beta_j^+ + \sum_{i=1}^{j-1} a_{ij} \beta_i^+ = \alpha_j$, we can further rewrite \cref{eq:coeffreduction} as
\begin{equation}\label{eq:coeffident}
\Delta_{c^m \omega_k}^{c^{m-1} \omega_k} = 
\prod_{j=1}^n \left( \Delta_{c \omega_j}^{\omega_j} \prod_{i < j}(\Delta_{c \omega_i}^{\omega_i})^{a_{ij}}\right)^{[c^{m-1} \beta_k^+:\alpha_j]}. 
\end{equation}
On the other hand, consider the antiautomorphism of $G$ which fixes $H$ and exchanges $x_i(t)$ and $x_{-i}(t)$.  This pulls back $\Delta_{\omega}^{\lambda}$ to $\Delta^{\omega}_{\lambda}$ and takes $G^{c,c^{-1}}$ to itself.  Applying it to \cref{eq:coeffident} we obtain
\[
\Delta^{c^m \omega_i}_{c^{m-1} \omega_i} = 
\prod_{j=1}^n \left( \Delta^{c \omega_j}_{\omega_j} \prod_{i < j}(\Delta^{c \omega_i}_{\omega_i})^{a_{ij}}\right)^{[c^{m-1} \beta_k^+:\alpha_j]}.
\]
Plugging these identities into the  $u = v = c^{m-1}s_1 \cdots s_{k-1}$ case of \Cref{prop:fundid} then yields the desired claim.
\end{proof}

\begin{proposition}
Let $G$ be a Kac-Moody group, $c$ a Coxeter element of its Weyl group, and $B_c$ the associated exchange matrix.  The coordinate ring of the double Bruhat cell $G^{c,c^{-1}}$ is an upper cluster algebra of type $B_c$ with coefficients.  The coordinate ring of the reduced double Bruhat cell $L^{c,c^{-1}}$ is an upper cluster algebra of type $B_c$ with principal coefficients.
\end{proposition}
\begin{proof}
\begin{enumerate}
\item Cite other papers.
\item Generalities on reducing double Bruhat cells and forgetting cefficients.
\item Redo Yang-Zelevinsky change of coefficients.
\end{enumerate}
\end{proof}

\begin{proposition}
Let $G$ be a Kac-Moody group, $c$ a Coxeter element of its Weyl group, and $B_c$ the associated exchange matrix.  All initial and preprojective cluster variables in the coordinate ring of $G^{c,c^{-1}}$ are restrictions of principal minors of highest weight representations.  Their $g$-vectors are of the form $\dotsc$, and the cluster variable with $g$-vector $g = (g_1,\dotsc,g_n)$ is the restriction of the principal minor of weight $\sum g_i \omega_i$.
\end{proposition}
\begin{proof}
\begin{enumerate}
\item Redo Yang-Zelevinsky computation of the exchange relations among preprojective minors.
\end{enumerate}
\end{proof}

\begin{proposition}
Let $G$ be a Kac-Moody group, $c$ a Coxeter element of its Weyl group, and $B_c$ the associated exchange matrix.  All postinjective cluster variables in the coordinate ring of $G^{c,c^{-1}}$ are restrictions of principal minors of lowest weight representations.  Their $g$-vectors are of the form $\dotsc$, and the cluster variable with $g$-vector $g = (g_1,\dotsc,g_n)$ is the restriction of the principal minor of weight $\sum g_i \omega_i$.
\end{proposition}
\begin{proof}
\begin{enumerate}
\item Compute change of coordinates between oppositely shuffled parametrizations.
\item Use this to show antifundamental minors are mutations of fundamental ones.
\item Think about coefficients in oppositely shuffled parametrization.
\item Show negative version of generalized minor identities reduce to exchange relations among preinjective minors.
\end{enumerate}
\end{proof}

\section{Level Zero Representations}

\begin{proposition}
Let $G = \widehat{LSL_{n+1}}$ be the affine Kac-Moody group of type $A_n^{(1)}$, $c$ a Coxeter element of the affine Weyl group, and $Q_c$ the associated affine quiver.  All regular cluster variables in the coordinate ring of $G^{c,c^{-1}}$ are principal minors of level zero representations.  
\end{proposition}
\begin{proof}
\begin{enumerate}
\item Given a regular representation $M_{[p,q]}$ compute the strand configuration whose minor has the same $F$-polynomial.
\item Show that the injective copresentation matches the weight of the minor.
\end{enumerate}
\end{proof}

\section{Examples in Other Types}

\begin{proposition}
\Cref{conj:mainconjecture} holds when $G$ is of type $B_2^{(1)}$, $B_3^{(1)}$, $C_2^{(1)}$, $D_4^{(1)}$, $G_2^{(1)}$ and $c$ is $\dotsc$. 
\end{proposition}
\begin{proof}
\begin{enumerate}
\item Write out the computation.
\end{enumerate}
\end{proof}

\section{Appendix: Double Bruhat Cells}


\section{Appendix: Cluster Algebras}
Let $A$ be any irreducible Cartan matrix of infinite type and let $c=s_1\dots s_n$ be a Coxeter element in the Weyl group of $A$. 
Because of the assumption, any prefix of the infinite word $c^\infty$ is reduced. 
The letters of $c^\infty$ (and thus its prefixes) are naturally labeled by pair of integers $(k,m)$ with $1\leq k\leq n$ and $m\geq 0$ where the element $c^ms_1\dots s_k$ corresponds to the indices $(k,m)$. 
\saySS{We should be able to do this also in finite type replacing prefixes with reduced words.}
Let $\cAb(B_c)$ be the cluster algebra with principal coefficients associated to $c$.
\saySS{We may want to settle the notation that relates $B_c$ to $A$ and $c$ onve and for all in some notation section.}

\begin{definition}
  The \emph{preprojective} cluster variables of $\cAb(B_c)$ are the cluster variables obtained by mutating along the word $c^\infty$.
  Namely $x_{k,m}$ is the cluster variable obtained from the initial cluster by mutating in the directions $(1,0)$ through $(k,m)$.
  \saySS{This phrasing sucks but I am fighting the blank page at the moment.}
\end{definition}

\begin{lemma}
  The preprojective cluster variables of $\cAb(B_c)$ satisfy the exchange relations
  \begin{equation}
    x_{k,m-1}x_{k,m} = \prod_{i<k}x_{i,m}^{-a_{i,k}} \prod_{i>k}x_{i,m-1}^{-a_{i,k}} + \prod_{j\in I}y_j^{[c^{m-1}\beta_k^+:\alpha_j]}
  \end{equation}
\end{lemma}

\begin{proposition}\label{prop:changeofcoeffs}
Let $B$ and $B'$ be $n+m \times n$ exchange matrices and $M$ and invertible $m \times m$ matrix just that the bottom $m \times n$ part of $B$ is equal to $M$ times the bottom $m \times n$ part of $B'$.  Let $x_1,\dotsc,x_n$ and $x'_1,\dotsc,x'_n$ be the mutable initial cluster variables of $\cA(B)$ and $\cA(B')$, respectively, and $y_1,\dotsc,y_m$, $y'_1,\dotsc, y'_m$ their frozen variables.  Then the map $x'_i \mapsto x_i$, $y'_j \mapsto \prod_{i=1}^m y_i^{M_{ij}}$ extends to an isomorphism of upper cluster algebras. 
\end{proposition}

\section{Appendix: Quiver Representations}
\subsection{Affine Tubes}
  Let $Q$ be an acyclic affine quiver of type $A$ with vertices labeled $\{1,\ldots,n\}$ counter-clockwise around $Q$.  Suppose $Q$ has $p$ clockwise pointing arrows and $q$ counter-clockwise pointing arrows.  Write $i_1,\ldots,i_k$ for the sinks and sources of $Q$ read counter-clockwise around the quiver and note that $k$ must be even.  For notational conevenience we denote $i_k$ by $i_0$ and assume that $i_1$ is a sink in $Q$.  Define full, extension-closed subcategories $\overrightarrow{\rep}Q$ and $\overleftarrow{\rep}Q$ of $\rep Q$ as follows:
  \begin{itemize}
    \item $\overrightarrow{\rep}Q$ is generated by the representations $M_{[i_{2t-1},i_{2t}]}$ for $1\le t\le k/2$ together with the representations $S_{j+i_{2t}}$ for $1\le t\le k/2$ and $1\le j\le i_{2t+1}-i_{2t}-1$;
    \item $\overleftarrow{\rep}Q$ is generated by the representations $M_{[i_{2t-2},i_{2t-1}]}$ for $1\le t\le k/2$ together with the representations $S_{j+i_{2t-1}}$ for $1\le t\le k/2$ and $1\le j\le i_{2t}-i_{2t-1}-1$.
  \end{itemize}
  Notice that $\overrightarrow{\rep}Q$ is generated by exactly $q$ indecomposable representations while $\overleftarrow{\rep}Q$ is generated by exactly $p$ indecomposable representations.
  \begin{proposition}\mbox{}
    \begin{enumerate}
      \item The modules listed above provide a complete set of simple objects in the categories $\overrightarrow{\rep}Q$ and $\overleftarrow{\rep}Q$.  Moreover, every object of $\overrightarrow{\rep}Q$ and $\overleftarrow{\rep}Q$ admits a unique Jordan-H\"older series with the corresponding simples as composition factors.
      \item Every regular representation of $Q$ is contained in one of $\overrightarrow{\rep}Q$ or $\overleftarrow{\rep}Q$.
      \item The regular exceptional representations of $Q$ are exactly the representations of $\overrightarrow{\rep}Q$ and $\overleftarrow{\rep}Q$ of length at most $q-1$ and $p-1$ respectively, where length refers to the Jordan-H\"older length in the respective categories.
    \end{enumerate}
  \end{proposition}
  \begin{proof}
    \begin{enumerate}
      \item Cite Ringel's ``Tame Algebras and Integral Quadratic Forms''?
    \end{enumerate}
  \end{proof}

  \begin{corollary}
    A representation $M$ of $Q$ is regular if and only if $\sum\limits_{\ell=1}^k\dim M_{i_k}$ is even.
  \end{corollary}

\section{Conventions}

\begin{itemize}

  \item 
    The Cartan matrix is $A=a_{i,j}$ and it's $n \times n$.

  \item 
    $I=[1,n]$ is the set of indices of the nodes of the Dynkin diagram. The labels are such that $c=s_1\dots s_n$.

  \item 
    $B_c$ is the matrix given by
    \[
      b_{ij} = \begin{cases}
        -a_{ij} & i<j\\
        a_{ij}  & i>j\\
        0       & i=j
      \end{cases}
    \]
    (negative below the diagonal)

  \item
    The Euler matrix $E_c$ is given by
    \[
      e_{ij} = \begin{cases}
        0       & i<j\\
        a_{ij}  & i>j\\
        1       & i=j
      \end{cases}
    \]
    (negative below the diagonal 0 above)

  \item
    The initial $B$-matrix for $G^{c,c^{-1}}$ with double reduced word $\bar 1,\dots,\bar n,n\dots 1$ is
    \[
      \left[
        \begin{array}{c}
          B_c\\
          E_{c^{-1}}\\
          E_{c^{-1}}
        \end{array}
      \right]
    \]
  
  \item 
    The minor $\Delta_\omega^\delta$ is our notation for YZ's $\Delta_{\delta, \omega}$.
  
  \item 
    $[\alpha:\alpha_j]$ notation defined immediately before \Cref{lem:minorsondbc}
  
  \item 
    The transpose $g \mapsto g^T$ of YZ is defined in passing without notation in the proof of \Cref{lem:minorsondbc}.
  
  \item 
    $L_{\omega_i}$ is the $i$th fundamental representation in the proof of \Cref{lem:minorsondbc}.

  \item 
    Used the notation $h^\omega$ for $\omega$ a weight and $h \in H$ without explaining it in the proof of \Cref{lem:minorsondbc}.

  \item 
    We use $x_{-i}(t)$ rather than a barred version for negative 1-parameter subgroups in the proof of \Cref{lem:minorsondbc}.

  \item 
    The identity $\beta_j^+ + \sum_{i=1}^{j-1} a_{ij} \beta_i^+ = \alpha_j$ is used in the proof of \Cref{lem:minorsondbc} without comment.
\end{itemize}

\end{document}
